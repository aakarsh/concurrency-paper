% Created 2014-04-04 Fri 19:09
\documentclass{article}
\usepackage[utf8]{inputenc}
\usepackage[T1]{fontenc}
\usepackage{fixltx2e}
\usepackage{graphicx}
\usepackage{longtable}
\usepackage{float}
\usepackage{wrapfig}
\usepackage{soul}
\usepackage{textcomp}
\usepackage{marvosym}
\usepackage{wasysym}
\usepackage{latexsym}
\usepackage{amssymb}
\usepackage{hyperref}
\tolerance=1000
\providecommand{\alert}[1]{\textbf{#1}}

\title{Synchronization in Linux}
\author{Aakarsh Nair}
\date{\today}

\begin{document}


\maketitle


\setcounter{tocdepth}{3}
\tableofcontents

\maketitle

\vspace*{1cm}
\section{Introduction}

\section{Linux Kernel Synchronization}


The linux kernel provides several synchronization primitives these include

\begin{itemize}
\item \textbf{Per-CPU variables}
  Use duplicate data structures for each cpu
\item \textbf{Atomic Operations}
  Attomically read-modify-write instruction to a counter
\item \textbf{Memory Barrier}
  Avoid instruction reordering
\item  \textbf{Spinlock}
  Lock with a busy wait
\item \textbf{Semaphores}
  Lock with blocking wait/sleep.
\item \textbf{Seqlocks}
  Lock based on access counter
\item \textbf{Local interrupt disabling}
  Forbid interrupt handling on a single CPU
\item \textbf{Local softirq disabling}
  Forbid deferrable function handling on a single CPU
\item \textbf{Read Copy Update}
  Lock free access to shared data structures using pointers
\end{itemize}


\subsection{Per-CPU Variables}

A per-CPU variable is an array of datastructures one per cpu. Each CPU
can read and modify its own elements without race. 

These are aligned in main memory so that each one falls in different
line of hardware cache. Concurrent access does not lead to cache line
snooping and invalidation.

Though per-cpu variables allow for data structures to be protected
against different CPUS they cannot be used to protect against
asynchronous accesses coming through the same CPU.

Accessing a per-CPU variable needs to be aware that premption might
cause the process to get migrated onto another CPU and thus it might
be advisable to turn off preemptions for per CPU variables.


\begin{center}
  \begin{tabular}{ l | l }
    
    \hline
    Function/Macors & Descriptions \\ \hline
    DEFINE\_PER\_CPU(type,name) & Statically Allocate a per-CPU called type \\ 
    per\_cpu(name,cpu) & Selects element for CPU cpu \\ 
    get\_cpu\_var(name) &  Disables kernel premeption \\
    put\_cpu\_var(name) &  Re-enables kernel premeption \\
    alloc\_percpu(ptr) &  dynamcally allocatee cpu variable \\
    free\_percpu(ptr) &  Release a dynamcally allocated cpu variable \\
    \hline
  \end{tabular}
\end{center}

\subsection{Atomic Operations}

Atomic read-modify-write instructions are executed atomically by a CPU
hardware via a single instruction which is executed without
interruptions by other CPUs.

\subsubsection{Atomic Operations 80x86}

\begin{itemize}
  \item Read-modify-write assembly instructions such as inc and dec
    that read data from memory and update it are atomic , provided
    stale data has not been read by another processor. This is the
    case in uniprocessor systems.

  \item Read-modify-write instructions whose opcode is prefixed by the
    \emph{lock byte} (\emph{0xf0}) are atomic on multiprocessor
    systems.  The lock byte will lock access to the mememory bus until
    the locking instruction finishes its operation.
  \item Instructions prefixed by the \emph{rep} byte \emph{0xf2,0xf3}
    which forces instructions to be repeated are not atomic since the
    CPU checks interrupts before each iteration.

\end{itemize}


\subsection{Memory Barrier}

Performance optimizations of compilers and CPUs play havoc with
synchronization primitives. These include compiler reordering
instructoins to optimize register usage, CPUs executing instructions
in parallel, reordering memory accesses.


\begin{itemize}

\item{Optimization barrier} Can be implemented using the
  \emph{barrier()} macro expanding to \emph{asm volatile"":::memory}

  Tells the compiler to insert empty assembly fragment. While the
  volatile keyword forbids the compiler from shuffling the
  instruction.  The memory keyword forces the compiler to use memory
  locations instead of those stored in the register. The CPU can still
  mix assembly instruction
  
\item{Memory barrier}

  In 80x86 list of serializing instructions which act as memory
  barriers:


  \begin{itemize}
    \item I/O port operations
    \item instructions with lock byte
    \item instructions affecting the IF flag in eflags register such as those instructions which write to registers
      \begin{itemize}
        \item control registers (cli)
        \item system  registers (sti)
        \item debug   registers          
      \end {itemize}
    \item Some instructions introduced in Pentium 4
      \begin{itemize}
        \item lfence - read barriers
        \item sfence - write barries
        \item mfence - read write barriers
      \end{itemize}
    \item Speial instructions - iret terminating interrupt or exception handler                  
  \end{itemize}
  
\end{itemize}

Read barriers maintain the serial order of read instructions, write
barriers maitain serial order of write instructions.

\begin{center}
  \begin{tabular}{ l | l }    
    \hline
    Function/Macors & Description \\ \hline
    mb() & Memory barrier for MP and UP \\ 
    rmb() & Read memory barrier for MP and UP  \\ 
    wmb() &  Write memory barrier for MP and UP \\
    smp\_mb() &  Memory barrier for MP only \\
    smp\_rmb() &  Read memory barrier for MP only \\
    smp\_wmb() &  Write memory barrier for MP only \\
    \hline
  \end{tabular}
\end{center}
  Macro expansions on  80x86

  \begin{center}
  \begin{tabular}{ l | l }    
    \hline
    Function/Macors & Description \\ \hline
    mb() &  \\ 
    rmb() & asm volatile ("lfence") or asm volatile ("lock;addl 0,0(\%\%esp)":::"memory")  \\ 
    wmb() & barrier()- intel never reorders write memory access only need an optimization barrier  \\
    smp\_mb() &   \\
    smp\_rmb() &  \\
    smp\_wmb() &  \\
    \hline
  \end{tabular}
\end{center}

All atomic operaitions act as memory barriers since they use the lock byte.

\subsection{Spinlock}




\subsection{Semaphores}


\section{MESI Cache Coherency Protocol}

The MESI protocol is a protocol used to implement cache and memory coherency amongst multiple CPUs. \cite{Birdetal2001}

Two bits are added to each ache line which represent the four states that a cache line can be in. These are

\begin{itemize}
\item Modified
  \begin{itemize}
    \item Cache Line is present only on current CPU
    \item Cache Line has been modified
    \item Write back needs to be performed
  \end{itemize}


\item Exclusive
    \begin{itemize}
    \item Cache Line is present only in current CPU
    \item Cache Line is cliean (matches main memory)
    \end{itemize}
    
  \item Shared
    \begin{itemize}
    \item Cache Line is present in multiple CPUs
    \item Cache Line is Clean (matches main memory)
    \end{itemize}
  \item Invalid 
\end{itemize}


\begin{thebibliography}{9}
\bibitem{mesi-wikipedia} 
  \textit{MESI protocol} 
  \url{http://en.wikipedia.org/wiki/MESI_protocol}
\bibitem{cache-coherency-primer} 
   \textit{Cache Coherency Primer}  \textit{Fabian “ryg” Giesen} \\
   \url{https://fgiesen.wordpress.com/2014/07/07/cache-coherency/}

 \bibitem{atomic-operations}
   \textit{Atomic Operations - CSE 378 University of Washington} \\
  \url{http://courses.cs.washington.edu/courses/cse378/07au/lectures/L25-Atomic-Operations.pdf}
\bibitem{memory-barriers}
  \textit{Linux Kernel Memory Barriers}
  \url{https://www.kernel.org/doc/Documentation/memory-barriers.txt}
\bibitem{columbia-sync-linux}
  \textit{Synchronization in Linux}
  \url{http://www.cs.columbia.edu/~junfeng/10sp-w4118/lectures/l11-synch-linux.pdf}

\bibitem{under-linux-kernel-sync}
  \textit{Understanding the Linux Kernel - Chapter 5} \\
  \textit{David Bovet and Marco Cesati}  

\bibitem{sadc}  
  \textit{Structures and Design of Computers } \\
  \textit{David E. Patterson and J.L. Hennessey} 
  

\end{thebibliography}

\end{document}
