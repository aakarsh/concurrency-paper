% Created 2014-04-04 Fri 19:09
\documentclass{article}
\usepackage[utf8]{inputenc}
\usepackage[T1]{fontenc}
\usepackage{fixltx2e}
\usepackage{graphicx}
\usepackage{longtable}
\usepackage{float}
\usepackage{wrapfig}
\usepackage{soul}
\usepackage{textcomp}
\usepackage{marvosym}
\usepackage{wasysym}
\usepackage{latexsym}
\usepackage{amssymb}
\usepackage{hyperref}
\tolerance=1000
\providecommand{\alert}[1]{\textbf{#1}}

\title{Linux Concurrency Implementation on x86}
\author{Aakarsh Nair}
\date{\today}

\begin{document}


\maketitle


\setcounter{tocdepth}{3}
\tableofcontents

\maketitle

\vspace*{1cm}
\section{Introduction}


\section{MESI Cache Coherency Protocol}

The MESI protocol is a protocol used to implement cache and memory coherency amongst multiple CPUs. \cite{Birdetal2001}

Two bits are added to each ache line which represent the four states that a cache line can be in. These are

\begin{itemize}
\item Modified
  \begin{itemize}
    \item Cache Line is present only on current CPU
    \item Cache Line has been modified
    \item Write back needs to be performed
  \end{itemize}


\item Exclusive
    \begin{itemize}
    \item Cache Line is present only in current CPU
    \item Cache Line is cliean (matches main memory)
    \end{itemize}
    
  \item Shared
    \begin{itemize}
    \item Cache Line is present in multiple CPUs
    \item Cache Line is Clean (matches main memory)
    \end{itemize}
  \item Invalid 
\end{itemize}


\begin{thebibliography}{9}
\bibitem{mesi-wikipedia} 
  \textit{MESI protocol} 
  \url{http://en.wikipedia.org/wiki/MESI_protocol}
\bibitem{cache-coherency-primer} 
   \textit{Cache Coherency Primer}  \textit{Fabian “ryg” Giesen} \\
   \url{https://fgiesen.wordpress.com/2014/07/07/cache-coherency/}

 \bibitem{atomic-operations}
   \textit{Atomic Operations - CSE 378 University of Washington} \\
  \url{http://courses.cs.washington.edu/courses/cse378/07au/lectures/L25-Atomic-Operations.pdf}
  

\end{thebibliography}

\end{document}
